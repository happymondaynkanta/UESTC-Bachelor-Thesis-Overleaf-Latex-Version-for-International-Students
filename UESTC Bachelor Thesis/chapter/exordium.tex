\documentclass{standalone}
% preamble: usepackage, etc.
\begin{document}

\thesischapterexordium

\textbf{\section{Overview and Background}}
Various machine vision applications, such as image recognition, identification and detection [1]–[10], use convolution neural networks as the main strategy. CNNs are increasingly being learnt on huge datasets and are being accelerated by increasingly better GPU machines, resulting in most advanced level performance as compared to conventional approaches. The popularity of CNN in machine vision can be attributed to two factors. To begin with, existing CNN-led solutions dominate various simple tasks, such as image super-resolution (SISR) [1], [2], [11], denoising [5], deblurring [12], compression[13], and reconstruction [6], by outperforming other approaches by a wide margin. Secondly, CNNs are also used as an extensiblecomponent that can be incorporated into traditional approaches, allowing them to be used more widely [12], [14], [15]. 


In this light, machine vision and image classification research has transitioned to deep learning approaches, which are regarded to be more dependable than classic machine learning techniques in terms of improving image recognition performance. \citeup{4}. The configuration of the human brain is made up of many units of neurons, which is the basis for which deep neural network is modeled after.


In this thesis, we utilize   deep learning frameworks to effectively process and extract audio and image data in order to obtain an effective classification and identification.  performance. We presented several comparison with stand-alone deep learning models and state-of-the-art methods.

\textbf{\section{Problem Statement}}

Recent technological development and increasing digital transmission of information have resulted in the creation of massive amounts of information. These information are nearly by definition multimedia of all formats which includes video, images, audio and so on. For most internet consumers, image and audio messages are the major format of correspondence, especially with the introduction of smart devices. 



\textbf{\section{Contribution and Significance}}

In conclusion, this dissertation examines the model construction and implementation of a multi-resolution discrete wavelet transform deep learning framework for obtaining good performance in image classification and identification. The data pre-processing aspect of this dissertation involves converting one-dimensional audio data into two-dimensional domain for effective training. There are two core model contribution is one of three major model contributions in this research; the first model is the multi-resolution analysis CNN framework. The second technique is the  multi-resolution analysis capsule network. The models are as follows:

\begin{enumerate}
    \item The capability of deep learning for image classification and identification. 
    
    
    \item An improved capsule network for accurately classifying images. This model is demonstrated by integrating wavelet into capsule network in a shared weighted fashion with continuous wavelet transform as a pre-processing technique to handle data dimensionality conversion. 
    
\end{enumerate}

\textbf{\section{Organization and Outline of the Dissertation}}

The remaining chapters of this dissertation are structured as follows:

\begin{enumerate}
    \item \textit{Chapter 2}: This chapter provides a useful overview of the existing literature 

    \item \textit{Chapter 3}: This chapter introduces the capability of deep learning for image classification and identification. The model's building blocks, design, and implementation, as well as enrich experimental analysis, are all thoroughly described.

    \item \textit{Chapter 4}: This chapter presents the results and discussion for accurately classifying images. 
    
    \item \textit{Chapter 5}: This chapter summarizes  and outlines the reported methodology and fundamental ideas of the thesis.
    
\end{enumerate}


\end{document}